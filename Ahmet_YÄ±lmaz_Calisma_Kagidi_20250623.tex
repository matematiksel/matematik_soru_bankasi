\documentclass[12pt,a4paper]{article}
\usepackage[utf8]{inputenc}
\usepackage[turkish]{babel}
\usepackage{enumitem}
\usepackage{geometry}
\usepackage{multicol}
\geometry{a4paper, margin=1in}

\setlist[enumerate,1]{label=\textbf{\arabic*.}}
\setlist[enumerate,2]{label={\textbf{\Alph*)}}, nolistsep}

\title{\textbf{Kişiye Özel Matematik Çalışma Kağıdı}}
\author{Ahmet Yılmaz}
\date{\today}

\begin{document}
\maketitle

\section*{Sorular}
\begin{enumerate}
\end{enumerate}
\begin{multicols}{2}
\begin{enumerate}[resume]
  \item 90 sayısının kaç tane pozitif tam sayı çarpanı vardır?
  \begin{enumerate}
    \item 8
    \item 10
    \item 12
    \item 14
  \end{enumerate}
  \vspace{1cm}
  \item 15 ile aralarında asal olan iki basamaklı en küçük doğal sayı kaçtır?
  \begin{enumerate}
    \item 10
    \item 11
    \item 12
    \item 13
  \end{enumerate}
  \vspace{1cm}
  \item Aşağıdakilerden hangisi 24 sayısının bir pozitif tam sayı çarpanı (böleni) DEĞİLDİR?
  \begin{enumerate}
    \item 4
    \item 6
    \item 9
    \item 12
  \end{enumerate}
  \vspace{1cm}
  \item Aşağıdaki sayılardan hangisi 10 ile aralarında asaldır?
  \begin{enumerate}
    \item 9
    \item 12
    \item 15
    \item 25
  \end{enumerate}
  \vspace{1cm}
\end{enumerate}
\end{multicols}
\newpage
\section*{Sorular ve Çözümleri (Orta)}
\begin{enumerate}
  \item 90 sayısının kaç tane pozitif tam sayı çarpanı vardır?
  \begin{enumerate}
    \item 8
    \item 10
    \item 12
    \item 14
  \end{enumerate}
  \vspace{5pt}
  \par\noindent\textbf{Çözüm:}
  \par\noindent 90'ı asal çarpanlarına ayıralım: $90 = 2^1 \cdot 3^2 \cdot 5^1$. Pozitif bölen sayısı, üslerin birer fazlasının çarpımıdır: $(1+1)(2+1)(1+1) = 2 \cdot 3 \cdot 2 = 12$.\vspace{1cm}

  \item 15 ile aralarında asal olan iki basamaklı en küçük doğal sayı kaçtır?
  \begin{enumerate}
    \item 10
    \item 11
    \item 12
    \item 13
  \end{enumerate}
  \vspace{5pt}
  \par\noindent\textbf{Çözüm:}
  \par\noindent 15'in asal çarpanları 3 ve 5'tir. Bir sayının 15 ile aralarında asal olması için 3'e ve 5'e bölünmemesi gerekir. İki basamaklı en küçük sayı 10'dur (5'e bölünür, olmaz). 11 hem 3'e hem 5'e bölünmez. Cevap 11.\vspace{1cm}

  \item Aşağıdakilerden hangisi 24 sayısının bir pozitif tam sayı çarpanı (böleni) DEĞİLDİR?
  \begin{enumerate}
    \item 4
    \item 6
    \item 9
    \item 12
  \end{enumerate}
  \vspace{5pt}
  \par\noindent\textbf{Çözüm:}
  \par\noindent 24 sayısının çarpanları 1, 2, 3, 4, 6, 8, 12, 24'tür. 9 bu listede yer almaz.\vspace{1cm}

  \item Aşağıdaki sayılardan hangisi 10 ile aralarında asaldır?
  \begin{enumerate}
    \item 9
    \item 12
    \item 15
    \item 25
  \end{enumerate}
  \vspace{5pt}
  \par\noindent\textbf{Çözüm:}
  \par\noindent İki sayının aralarında asal olması için 1'den başka ortak bölenleri olmamalıdır. 10'un asal çarpanları 2 ve 5'tir. Bir sayının 10 ile aralarında asal olması için 2'ye ve 5'e bölünmemesi gerekir. Şıklardan sadece 9 bu kurala uyar.\vspace{1cm}

\end{enumerate}
\end{document}

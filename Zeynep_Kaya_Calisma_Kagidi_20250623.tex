\documentclass[12pt,a4paper]{article}
\usepackage[utf8]{inputenc}
\usepackage[turkish]{babel}
\usepackage{enumitem}
\usepackage{geometry}
\usepackage{multicol}
\geometry{a4paper, margin=1in}

\setlist[enumerate,1]{label=\textbf{\arabic*.}}
\setlist[enumerate,2]{label={\textbf{\Alph*)}}, nolistsep}

\title{\textbf{Kişiye Özel Matematik Çalışma Kağıdı}}
\author{Zeynep Kaya}
\date{\today}

\begin{document}
\maketitle

\section*{Sorular}
\begin{enumerate}
  \item Bir okuldaki öğrenciler 6'şarlı veya 8'erli gruplandırıldığında her seferinde 2 öğrenci artmaktadır. Okuldaki öğrenci sayısının 100'den fazla olduğu bilindiğine göre, en az kaç öğrenci vardır?
  \begin{enumerate}
    \item 118
    \item 120
    \item 122
    \item 124
  \end{enumerate}
  \vspace{1cm}
\end{enumerate}
\newpage
\section*{Sorular ve Çözümleri (Orta)}
\begin{enumerate}
  \item Bir okuldaki öğrenciler 6'şarlı veya 8'erli gruplandırıldığında her seferinde 2 öğrenci artmaktadır. Okuldaki öğrenci sayısının 100'den fazla olduğu bilindiğine göre, en az kaç öğrenci vardır?
  \begin{enumerate}
    \item 118
    \item 120
    \item 122
    \item 124
  \end{enumerate}
  \vspace{5pt}
  \par\noindent\textbf{Çözüm:}
  \par\noindent 6 ve 8'in en küçük ortak katı (EKOK) bulunur. EKOK(6,8) = 24. Öğrenci sayısı 24'ün bir katından 2 fazladır. 24'ün katları: 24, 48, 72, 96, 120... 100'den büyük en küçük kat 120'dir. Öğrenci sayısı = 120 + 2 = 122.\vspace{1cm}

\end{enumerate}
\end{document}
